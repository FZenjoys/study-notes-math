%\documentclass[11pt]{article}
%\usepackage{geometry}                % See geometry.pdf to learn the layout options. There are lots.
%\geometry{a4paper}                   % ... or a4paper or a5paper or ... 
%\usepackage{graphicx}
%\usepackage{amssymb}
%\usepackage{epstopdf}
%\usepackage{color}
%\DeclareGraphicsRule{.tif}{png}{.png}{`convert #1 `dirname #1`/`basename #1 .tif`.png}

%\title{Euler Angles to Quaternions}
%\author{Basile Graf}
%%\date{}                                           % Activate to display a given date or no date

%\begin{document}
%\maketitle
%%\section{}
%%\subsection{}



\section{欧拉角到四元数}
\label{sec_Eul2Quat}

每个轴绕欧拉角的三个旋转可以被写为 \\

$$ R_{\psi} =  \left[ \begin {array}{ccc} \cos \left( {\it \psi} \right) &-\sin \left( {\it \psi} \right) &0\\\noalign{\medskip}\sin \left( {\it \psi} \right) &\cos \left( {\it \psi} \right) &0\\\noalign{\medskip}0&0&1\end {array} \right] $$\\

$$ R_{\theta} =  \left[ \begin {array}{ccc} \cos \left( {\it \theta} \right) &0&\sin \left( {\it \theta} \right) \\\noalign{\medskip}0&1&0\\\noalign{\medskip}-\sin \left( {\it \theta} \right) &0&\cos \left( {\it \theta} \right) \end {array} \right] $$\\


$$ R_{\phi} =  \left[ \begin {array}{ccc} 1&0&0\\\noalign{\medskip}0&\cos \left( {\it \phi} \right) &-\sin \left( {\it \phi} \right) \\\noalign{\medskip}0&\sin \left( {\it \phi} \right) &\cos \left( {\it \phi} \right) \end {array} \right] $$\\

它们结合在一起定义了旋转矩阵 \\

$$ R = R_{\phi} R_{\theta} R_{\psi}. $$\\ 

这三个旋转也可以表示为四元数旋转 \\

\begin{center}
$ \mathbf{q_{\phi}} =  \left[ \begin {array}{c} \cos \left( \frac{1}{2}\,{\it \phi} \right) \\\noalign{\medskip}\sin \left( \frac{1}{2}\,{\it \phi} \right) \\\noalign{\medskip}0\\\noalign{\medskip}0\end {array} \right] $ $\qquad$ $ \mathbf{q_{\theta}} =  \left[ \begin {array}{c} \cos \left( \frac{1}{2}\,{\it \theta} \right) \\\noalign{\medskip}0\\\noalign{\medskip}\sin \left( \frac{1}{2}\,{\it \theta} \right) \\\noalign{\medskip}0\end {array} \right] $ $\qquad$ $ \mathbf{q_{\psi}} =  \left[ \begin {array}{c} \cos \left( \frac{1}{2}\,{\it \psi} \right) \\\noalign{\medskip}0\\\noalign{\medskip}0\\\noalign{\medskip}\sin \left( \frac{1}{2}\,{\it \psi} \right) \end {array} \right] .$\\
\end{center}

\vspace{0.8cm}
然后把这三个数相乘就可以得到四元数 \\

$$ \mathbf{q} = \mathbf{q_{\phi}} \circ \mathbf{q_{\theta}} \circ \mathbf{q_{\psi}} =  \left[ \begin {array}{c} \cos \left( \frac{1}{2}\,{\it \phi} \right) \cos \left( \frac{1}{2}\,{\it \theta} \right) \cos \left( \frac{1}{2}\,{\it \psi} \right) -\sin \left( \frac{1}{2}\,{\it \phi} \right) \sin \left( \frac{1}{2}\,{\it \theta} \right) \sin \left( \frac{1}{2}\,{\it \psi} \right) \\\noalign{\medskip}\cos \left( \frac{1}{2}\,{\it \psi} \right) \cos \left( \frac{1}{2}\,{\it \theta} \right) \sin \left( \frac{1}{2}\,{\it \phi} \right) +\cos \left( \frac{1}{2}\,{\it \phi} \right) \sin \left( \frac{1}{2}\,{\it \theta} \right) \sin \left( \frac{1}{2}\,{\it \psi} \right) \\\noalign{\medskip}\cos \left( \frac{1}{2}\,{\it \psi} \right) \cos \left( \frac{1}{2}\,{\it \phi} \right) \sin \left( \frac{1}{2}\,{\it \theta} \right) -\cos \left( \frac{1}{2}\,{\it \theta} \right) \sin \left( \frac{1}{2}\,{\it \phi} \right) \sin \left( \frac{1}{2}\,{\it \psi} \right) \\\noalign{\medskip}\cos \left( \frac{1}{2}\,{\it \phi} \right) \cos \left( \frac{1}{2}\,{\it \theta} \right) \sin \left( \frac{1}{2}\,{\it \psi} \right) +\cos \left( \frac{1}{2}\,{\it \psi} \right) \sin \left( \frac{1}{2}\,{\it \phi} \right) \sin \left( \frac{1}{2}\,{\it \theta} \right) \end {array} \right]. $$ \\ 




注意:此结果取决于欧拉角和旋转轴的顺序和选择中使用的约定!参见文档\cite{mathworksQuat}


%\end{document}  
