\documentclass[border=2pt]{standalone}
\usepackage{tikz}
\usepackage{tikz-3dplot}
\usepackage{amsmath}
\usepackage{amssymb}

\begin{document}

%Angle Definitions
%-----------------

%set the plot display orientation
%synatax: \tdplotsetdisplay{\theta_d}{\phi_d}
\tdplotsetmaincoords{60}{115}

%define polar coordinates for some vector
%TODO: look into using 3d spherical coordinate system
\pgfmathsetmacro{\rvec}{1.0}
\pgfmathsetmacro{\thetavec}{60}
\pgfmathsetmacro{\phivec}{60}

%start tikz picture, and use the tdplot_main_coords style to implement the display 
%coordinate transformation provided by 3dplot
\begin{tikzpicture}[scale=6,tdplot_main_coords]

%set up some coordinates 
%-----------------------
\coordinate (O) at (0,0,0);
\coordinate (I) at (0,0,1);

%determine a coordinate (P) using (r,\theta,\phi) coordinates.  This command
%also determines (Pxy), (Pxz), and (Pyz): the xy-, xz-, and yz-projections
%of the point (P).
%syntax: \tdplotsetcoord{Coordinate name without parentheses}{r}{\theta}{\phi}
\tdplotsetcoord{P} {\rvec}{\thetavec+3.5}{\phivec}
\tdplotsetcoord{P1}{\rvec}{\thetavec+9.5}{\phivec-20}
\tdplotsetcoord{P2}{\rvec}{\thetavec+15} {\phivec-20}

%draw figure contents
%--------------------

%draw the main coordinate system axes
\draw[->,>=latex] (0,0,0) -- ( 1.2, 0  , 0  ) ;
\draw[->,>=latex] (0,0,0) -- ( 0  , 1.2, 0  ) ;
\draw[->,>=latex] (0,0,0) -- ( 0  , 0  , 1.2) ;
\node[left] at (I) {$\boldsymbol{I}$}; 

%draw a vector from origin to point (P) 
%\draw[->,>=latex] (O) -- (P);
%\draw[->,>=latex] (O) -- (P1);

%draw projection on xy plane, and a connecting line
%\draw[dashed] (O) -- (Pxy);
%\draw[dashed] (P) -- (Pxy);

%draw the angle \phi, and label it
%syntax: \tdplotdrawarc[coordinate frame, draw options]{center point}{r}{angle}{label options}{label}
%\tdplotdrawarc{(O)}{0.2}{0}{\phivec}{anchor=north}{$\phi$}

%set the rotated coordinate system so the x'-y' plane lies within the
%"theta plane" of the main coordinate system
%syntax: \tdplotsetthetaplanecoords{\phi}
\tdplotsetthetaplanecoords{\phivec}

%draw theta arc and label, using rotated coordinate system
%\tdplotdrawarc[tdplot_rotated_coords]{(0,0,0)}{0.3}{30}{\thetavec}{anchor=south west}{$\frac{\theta}{2}$}

%draw some dashed arcs, demonstrating direct arc drawing
\draw[dashed,tdplot_rotated_coords] (\rvec,0,0) arc (0:90:\rvec);
\draw[dashed] (\rvec,0,0) arc (0:90:\rvec);
\draw[very thick,tdplot_rotated_coords,<-,>=latex,color=orange] (P) arc (\thetavec+3.5:0:\rvec);

%\draw [very thin, tdplot_rotated_coords, lightgray!10,inner color=lightgray!100,outer color=lightgray!10] (P1) +(-0.01,-0.01,-0.01)  circle (0.15);
%\draw [very thin, tdplot_rotated_coords, lightgray!10,inner color=lightgray!100,outer color=lightgray!10] (P1) ++(-0.01,-0.01,-0.01)  ellipse (0.18 and 0.16);

%\draw[very thick,tdplot_rotated_coords,<-,>=latex,color=red ] (P1) arc (30:\thetavec:\rvec);
\tdplotsetrotatedcoords{0}{-90}{0};
\draw[dashed, tdplot_rotated_coords] (\rvec,0,0) arc (0:90:\rvec);
\tdplotsetrotatedcoords{-90}{-90}{0};
\draw[dashed, tdplot_rotated_coords] (\rvec,0,0) arc (0:90:\rvec);
\tdplotsetrotatedcoords{-90}{210}{0};
\draw[dashed, tdplot_rotated_coords] (\rvec,0,0) arc (0:90:\rvec);
\draw[very thick,tdplot_rotated_coords,->,>=latex,color=blue] (P) arc (27:46:\rvec);
%\draw[very thick,tdplot_rotated_coords,<-,>=latex,color=orange] (P) arc (15:0:\rvec);

\tdplotsetthetaplanecoords{\phivec-20}
\draw[dashed,tdplot_rotated_coords] (\rvec,0,0) arc (0:90:\rvec);
\draw[very thick,tdplot_rotated_coords,<-,>=latex,color=red] (P1) arc (\thetavec+9.5:0:\rvec);

%set the rotated coordinate definition within display using a translation
%coordinate and Euler angles in the "z(\alpha)y(\beta)z(\gamma)" euler rotation convention
%syntax: \tdplotsetrotatedcoords{\alpha}{\beta}{\gamma}
%\tdplotsetrotatedcoords{\phivec}{\thetavec}{0}

%translate the rotated coordinate system
%syntax: \tdplotsetrotatedcoordsorigin{point}
%\tdplotsetrotatedcoordsorigin{(P)}

%use the tdplot_rotated_coords style to work in the rotated, translated coordinate frame
%\draw [very thin, tdplot_rotated_coords, gray] ( 0.5-1.2, 0  ,1.2) -- (-0.5-1.2, 0  ,1.2);
%\draw [very thin, tdplot_rotated_coords, gray] ( 0  -1.2, 0.5,1.2) -- ( 0  -1.2,-0.5,1.2);
%\draw [very thin, tdplot_rotated_coords, gray] ( 0.6-1.2, 0.6,1.2) -- ( 0.6-1.2,-0.6,1.2);
%\draw [very thin, tdplot_rotated_coords, gray] ( 0.6-1.2, 0.6,1.2) -- (-0.6-1.2, 0.6,1.2);
%\draw [very thin, tdplot_rotated_coords, gray] (-0.6-1.2,-0.6,1.2) -- ( 0.6-1.2,-0.6,1.2);
%\draw [very thin, tdplot_rotated_coords, gray] (-0.6-1.2,-0.6,1.2) -- (-0.6-1.2, 0.6,1.2);

%\draw [very thin, tdplot_rotated_coords, lightgray!10,inner color=lightgray!100,outer color=lightgray!10] (-0.4-1.2,0,1.2)  ellipse (0.18 and 0.16);\draw[very thick, tdplot_rotated_coords, ->,>=latex] (0-1.2,0,1.2) -- (-0.4-1.2,0,1.2) node[anchor=south west]{$\boldsymbol{\delta}$};
%\draw[fill = white, tdplot_rotated_coords] (0-1.2,0,1.2) node[below right] {$\bar{\boldsymbol{q}}_{n}$} circle (0.5pt);
%\draw[tdplot_rotated_coords] (0.4-1.2,0,1.2) node[right, rotate=19] {tangent space};

\draw[fill = white] (P)  circle (0.5pt) node[anchor=south east] {$\boldsymbol{\mathcal{X}}_{1}$};
\draw[fill = white] (P1) circle (0.5pt) node[anchor=south east] {$\boldsymbol{\mathcal{X}}_{2}$};
\draw {(P)} +(-0.35,-0.40,0) node[above right] {$\boldsymbol{I} \oplus \boldsymbol{\theta}_{1}$};
\draw {(P1)} +(-0.08,-0.01,0) node[above right] {$\boldsymbol{\mathcal{X}}_{1} \oplus \boldsymbol{\theta}_{2}$};
%\draw[fill = white] (P2) circle (0.5pt) node[anchor=west] {$\boldsymbol{q}_{n}$};

%\draw [very thin, tdplot_rotated_coords]  (P)  circle (0.2);

\end{tikzpicture}

\end{document}

