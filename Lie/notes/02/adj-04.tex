\documentclass[border=2pt]{standalone}
\usepackage{tikz}
\usetikzlibrary{quotes,angles}
\usepackage{amsmath}
\usepackage{amssymb}

\begin{document}

\begin{tikzpicture}[scale=5]

% Draw x and y axis lines
\draw [->,>=latex] (-1.2,0) -- (1.2,0) node [below] {$\mathbf{q}$};
\draw [->,>=latex] (0,-0.2) -- (0,1.2) node [left] {$w$};
\node [above] at (0, 1.25) {North};
\node [below] at (0,-0.25) {South};
\node[above left] at (-0.7,0.7) {$\mathrm{S}^{3}$};
%\node[above] at (0.0,1.40) {$\mathrm{S}^{3}$ : The unit quaternion numbers};

% Draw a arc at the origin of radius 1
\draw (1,0) arc (000:180:1);
\filldraw[black] (0,1) circle (0.5pt) node[above left ] {$\mathcal{E}$} ;

\filldraw[black] (0.5000,0.8660) circle (0.5pt) node[above] {$\mathcal{X}$} ;
\filldraw[black] (0.8660,0.5000) circle (0.5pt) node[left ] {$\mathcal{Y} = \mathcal{X} \boxplus \, {}^\mathcal{X}\boldsymbol{\tau}$} ;

\begin{scope} [rotate=-30]

\draw (-0.5000,1) -- ( 0.6000,1)  node[right,rotate=-30] {tangent space} ;
\draw [->,>=latex, very thick, blue] (0,1) -- ( 0.5236,1)  node[above,rotate=-30] {${}^\mathcal{X}\boldsymbol{\tau}^{\wedge}$} ;
\draw [->,>=latex, very thick, red] (0,1) arc (90:60:1);
\draw [->,>=latex] [very thin] (0.5236,0.99) arc (00:-11:0.5236) ;
\node [right,rotate=-30] at (0.5236,0.91) {\tiny$\exp\left(\color{blue}{}^\mathcal{X}\boldsymbol{\tau}^{\wedge}\color{black}\right)$} ;

\end{scope}

%%%%%%%%%%%%%%%%%%%%%%%%%%%%%%%%%%%%%%%%%%%%%%%%%%%%%%%%%%%%%%%%%%%%%%%
\begin{scope} [xshift=80]

% Draw x and y axis lines
\draw [->,>=latex] (-1.2,0) -- (1.2,0) node [below] {$\mathbf{q}$};
\draw [->,>=latex] (0,-0.2) -- (0,1.2) node [left] {$w$};
\node [above] at (0, 1.25) {North};
\node [below] at (0,-0.25) {South};
\node[above left] at (-0.7,0.7) {$\mathrm{S}^{3}$};
%\node[above] at (0.0,1.40) {$\mathrm{S}^{3}$ : The unit quaternion numbers};

% Draw a arc at the origin of radius 1
\draw (1,0) arc (000:180:1);
\filldraw[black] (0,1) circle (0.5pt) node[above left ] {$\mathcal{E}$} ;

\begin{scope} [rotate=30]

\draw [->,>=latex] (-1.2,0) -- (1.2,0) node [below,rotate=30] {$\mathbf{q}^\prime$};
\draw [->,>=latex] (0,-0.2) -- (0,1.2) node [left ,rotate=30] {$w^\prime$};

% Draw a arc at the origin of radius 1
\draw (1,0) arc (000:180:1);
\filldraw[black] (0,1) circle (0.5pt) node[above left ,rotate=30] {$\mathcal{E}^\prime$} ;

\end{scope}

\filldraw[black] (0.5000,0.8660) circle (0.5pt) node[right] {$\mathcal{Y} = \mathcal{X}^\prime = {}^\mathcal{E}\boldsymbol{\tau} \boxplus \mathcal{X}$} ;
%\filldraw[black] (0.8660,0.5000) circle (0.5pt) node[left ] {$\mathcal{Y}$} ;

\draw (-0.55,1) -- ( 0.55,1)  node[right] {tangent space} ;
\draw [->,>=latex, very thick, blue] (0,1) -- (-0.5236,1)  node[above] {${}^\mathcal{E}\boldsymbol{\tau}^{\wedge}$} ;
\draw [->,>=latex, very thick, red] (0,1) arc (090:120:1);
\draw [->,>=latex] [very thin] (-0.5236,0.99) arc (00:12:-0.5236) ;
\node [right] at (-0.5236,0.94) {\tiny$\exp\left(\color{blue}{}^\mathcal{X}\boldsymbol{\tau}^{\wedge}\color{black}\right)$} ;

\end{scope}

%%%%%%%%%%%%%%%%%%%%%%%%%%%%%%%%%%%%%%%%%%%%%%%%%%%%%%%%%%%%%%%%%%%%%%%

\node[below right,scale=1.2] at (1.0,1.6) {The adjoint};

\node[below right] at (0.5,1.5) {$
\boxed{
\begin{array}{rl}
\mathrm{Exp}\left(^{\mathcal{E}}\boldsymbol{\tau}\right)\mathcal{X} & =\mathcal{X}\mathrm{Exp}\left(^{\mathcal{X}}\boldsymbol{\tau}\right)\\
\exp\left(^{\mathcal{E}}\boldsymbol{\tau}^{\wedge}\right) & =\mathcal{X}\exp\left(^{\mathcal{X}}\boldsymbol{\tau}^{\wedge}\right)\mathcal{X}^{-1}=\exp\left(\mathcal{X}\:{}^{\mathcal{X}}\boldsymbol{\tau}^{\wedge}\:\mathcal{X}^{-1}\right)\\
^{\mathcal{E}}\boldsymbol{\tau}^{\wedge} & =\mathcal{X}\:{}^{\mathcal{X}}\boldsymbol{\tau}^{\wedge}\:\mathcal{X}^{-1}
\end{array} 
}
$} ;

\draw [->] [very thin] (1.01,0.70) to [out= 60,in=-90] (1.22,1.20) ;
\draw [->] [very thin] (2.22,1.05) to [out=150,in=-60] (0.86,1.20) ;


\end{tikzpicture}

\end{document}

