\documentclass[border=2pt]{standalone}
\usepackage{tikz}
\usepackage{tikz-3dplot}
\usepackage{amsmath}
\usepackage{amssymb}

\begin{document}

%Angle Definitions
%-----------------
%
%set the plot display orientation
%synatax: \tdplotsetdisplay{\theta_d}{\phi_d}
\tdplotsetmaincoords{60}{110}
%
%define polar coordinates for some vector
%TODO: look into using 3d spherical coordinate system
\pgfmathsetmacro{\rvec}{1.0}
\pgfmathsetmacro{\thetavec}{30}
\pgfmathsetmacro{\phivec}{60}
%
%start tikz picture, and use the tdplot_main_coords style to implement the display coordinate transformation provided by 3dplot
\begin{tikzpicture}[scale=4,tdplot_main_coords]

\node[above] at (-0.25,-0.25) {$\{A\}$};
\node[above] at (-0.25,+0.25,+0.75) {$\{B\}$};

%set up some coordinates 
%-----------------------
\coordinate (O) at (0,0,0);

% O^{\prime}
%determine a coordinate (OP) using (r,\theta,\phi) coordinates.  This command also determines (OPxy), (OPxz), and (OPyz): the xy-, xz-, and yz-projections of the point (OP).
%synatax: \tdplotsetcoord{Coordinate name without parentheses}{r}{\theta}{\phi}
\tdplotsetcoord{OP}{\rvec}{\thetavec}{\phivec}

%draw figure contents
%--------------------

%draw the main coordinate system axes
\draw[thick,->,>=latex] (0,0,0) -- ( 1.2, 0  , 0  ) node[anchor=north east]{$x$};
\draw[thick,->,>=latex] (0,0,0) -- ( 0  , 1.2, 0  ) node[anchor=north west]{$y$};
\draw[thick,->,>=latex] (0,0,0) -- ( 0  , 0  , 1.2) node[anchor=south]     {$z$};

%draw a vector from origin to point (OP) 
\draw node[left,color=black]{$O$} [thick,->,>=latex,color=red] (O) -- node[anchor=north west,color=black]{$\mathbf{r}$} (OP) node[left,color=black]{$O^{\prime}$} ;

%draw projection on xy plane, and a connecting line
\draw[thin, dashed, color=red] (O) -- (OPxy);
\draw[thin, dashed, color=red] (OP) -- (OPxy);

%draw the angle \phi, and label it
%syntax: \tdplotdrawarc[coordinate frame, draw options]{center point}{r}{angle}{label options}{label}
%\tdplotdrawarc{(O)}{0.2}{0}{\phivec}{anchor=north}{$\phi$}


%set the rotated coordinate system so the x'-y' plane lies within the "theta plane" of the main coordinate system
%syntax: \tdplotsetthetaplanecoords{\phi}
\tdplotsetthetaplanecoords{\phivec}

%draw theta arc and label, using rotated coordinate system
%\tdplotdrawarc[tdplot_rotated_coords]{(0,0,0)}{0.4}{0}{\thetavec}{anchor=south west}{$\theta$}

%draw some dashed arcs, demonstrating direct arc drawing
\draw[dashed,tdplot_rotated_coords] (\rvec,0,0) arc (0:90:\rvec);
\draw[dashed] (\rvec,0,0) arc (0:90:\rvec);

%set the rotated coordinate definition within display using a translation coordinate and Euler angles in the "z(\alpha)y(\beta)z(\gamma)" euler rotation convention
%syntax: \tdplotsetrotatedcoords{\alpha}{\beta}{\gamma}
\tdplotsetrotatedcoords{\phivec}{\thetavec}{0}

%translate the rotated coordinate system
%syntax: \tdplotsetrotatedcoordsorigin{point}
\tdplotsetrotatedcoordsorigin{(OP)}

%use the tdplot_rotated_coords style to work in the rotated, translated coordinate frame
\draw[thick,tdplot_rotated_coords,->,>=latex] (0,0,0) -- (0.6,0.0,0.0) node[anchor=north west]{$x^{\prime}$};
\draw[thick,tdplot_rotated_coords,->,>=latex] (0,0,0) -- (0.0,0.6,0.0) node[anchor=west]{$y^{\prime}$};
\draw[thick,tdplot_rotated_coords,->,>=latex] (0,0,0) -- (0.0,0.0,0.6) node[anchor=south]{$z^{\prime}$};

%WARNING:  coordinates defined by the \coordinate command (eg. (O), (OP), etc.) cannot be used in rotated coordinate frames.  Use only literal coordinates.  

%draw some vector, and its projection, in the rotated coordinate frame
\draw[thick,->,>=latex,color=blue,tdplot_rotated_coords] (0.0,0.0,0.0) -- (0.3,0.3,0.3) node[right,color=black]{$P^{\prime}$} ;
\draw[thin, dashed,color=blue,tdplot_rotated_coords]     (0.0,0.0,0.0) -- (0.3,0.3,0.0);
\draw[thin, dashed,color=blue,tdplot_rotated_coords]     (0.3,0.3,0.0) -- (0.3,0.3,0.3);

%show its phi arc and label
%\tdplotdrawarc[tdplot_rotated_coords,color=blue]{(0,0,0)}{0.2}{0}{45}{anchor=north west,color=black}{$\phi'$}

%change the rotated coordinate frame so that it lies in its theta plane.  Note that this overwrites the original rotated coordinate frame
%syntax: \tdplotsetrotatedthetaplanecoords{\phi'}
\tdplotsetrotatedthetaplanecoords{45}

%draw theta arc and label
%\tdplotdrawarc[tdplot_rotated_coords,color=blue]{(0,0,0)}{0.2}{0}{55}{anchor=south west,color=black}{$\theta'$}

\end{tikzpicture}

\end{document}

