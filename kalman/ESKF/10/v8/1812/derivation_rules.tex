% !TEX root = micro_Lie_theory.tex

%%%%%%%%%%%%%%%%%%%%%%%%%%%%%%%%%%%%%%%%%%%%%%%%%%%%%%%%%%%%
\section{流形上的微分法则}
\label{sec:derivatives_M}

对于我们所使用的所有经典流形 $\cM$ ,对于求逆(\emph{inversion})、组合(\emph{composition})、求幂(\emph{exponentiation})和作用(\emph{action})的初等Jacobian矩阵,我们可以确定封闭形式。
此外,其中一些形式可能与伴随 $\Ad[\cM]{\cX}$ 有关,后者成为微分过程的中心块。
$\Log$、 $\op$ 和 $\om$ 等其它形式可以很容易地从它们当中推导出来。
一旦找到这些形式或“块”,所有其它Jacobian矩阵都遵循链式法则。
除了我们下面介绍的所谓的左Jacobian矩阵(\emph{left Jacobian}),这里扩展的所有Jacobian矩阵都是右Jacobian矩阵(\emph{right-Jacobian}),即由方程 \eqRef{equ:Jacobian} 所定义。
通过遵循这里的提示,感兴趣的读者应该不会发现在扩展左Jacobian矩阵时有什么特别的困难。
对于不愿意这样做的读者,方程 \eqRef{equ:derivatives_lr_adjoints} 可用于此目的,因为


%
\begin{align}
\ldpar{f(\cX)}{\cX} = \Ad[\cN]{f(\cX)}\rdpar{f(\cX)}{\cX} \Ad[\cM]{\cX}\inv
~.
\end{align}


我们使用符号 $\mjac{f(\cX)}{\cX}\te\ndpar{f(\cX)}{\cX}$ 和 $\mjac{\cY}{\cX}\te\ndpar{\cY}{\cX}$ 。
我们还注意到, $\Ad[\cM]{\cX}\inv$ 应该由 $\Ad[\cM]{\cX\inv}$ 实现 ---参见方程 \eqssRef{equ:Adj5,equ:Adj7} 以及下面的注释。



\subsection{链式法则}
\label{sec:jacs_chain_rule}

对于 $\cY=f(\cX)$ 和 $\cZ=g(\cY)$ 我们有 $\cZ=g(f(\cX))$。
链式法则简单地说,
%
\begin{align}
\ndpar{\cZ}{\cX} = \ndpar{\cZ}{\cY}\,\ndpar{\cY}{\cX}
\qquad \textrm{or} \qquad
\mjac{\cZ}{\cX} = \mjac{\cZ}{\cY}\,\mjac{\cY}{\cX}
~.
\end{align}
%
我们为了右Jacobian矩阵在这里证明它,其中使用方程 \eqRef{equ:lin_approx} 三次,
%
\begin{align*}
g(f(\cX))\op\mjac{\cZ}{\cX}\bftau \gets g(f(\cX\op\bftau)) &\to g(f(\cX)~\op\mjac{\cY}{\cX}\bftau) 
  \\
  &\to g(f(\cX))\op\mjac{\cZ}{\cY}\mjac{\cY}{\cX}\bftau
\end{align*}
%
其中箭头表示极限为 $\bftau\to0$ ,因此 $\mjac{\cZ}{\cX}=\mjac{\cZ}{\cY}\mjac{\cY}{\cX}$ 。
左Jacobian矩阵和交叉Jacobian矩阵的证明类似,分别使用方程 \eqssRef{equ:lin_approx_left,equ:lin_approx_rl,equ:lin_approx_lr} 。
%
注意,当混合右、左和交叉Jacobian矩阵时,我们还需要链接参考坐标系,如
%
\begin{align}
\rldpar{\cZ}{\cX}
  &= \rrdpar{\cZ}{\cY}\,\rldpar{\cY}{\cX}
  = \rldpar{\cZ}{\cY}\,\lldpar{\cY}{\cX} \label{equ:chain_rule_cross_1}
\\
\lrdpar{\cZ}{\cX}
  &= \lrdpar{\cZ}{\cY}\,\rrdpar{\cY}{\cX}
  = \lldpar{\cZ}{\cY}\,\lrdpar{\cY}{\cX}\label{equ:chain_rule_cross_2}
~,
\end{align}
%
其中方程 \eqRef{equ:chain_rule_cross_1} 的第一个特征的证明通过写方程,
%
\begin{align*}
g(f({^\cE\!\bftau}\op\cX)) 
  &\xrightarrow[{^\cE\!\bftau}\to0]{\eqRef{equ:lin_approx_lr}} 
  g(f(\cX))\op\rldpar{\cZ}{\cX}\,{^\cE}\!\bftau
  ~; \\ 
g(f({^\cE\!\bftau}\op\cX)) 
  &\xrightarrow[{^\cE\!\bftau}\to0]{\eqRef{equ:lin_approx_lr}} 
  g\left(f(\cX)~\op\rldpar{\cY}{\cX}\,{^\cE}\!\bftau\right) \to
  \\
  &\xrightarrow[{^\cE\!\bftau}\to0]{\eqRef{equ:lin_approx}} 
  g(f(\cX))\op\rrdpar{\cZ}{\cY}\,\rldpar{\cY}{\cX}\,{^\cE}\!\bftau 
~,
\end{align*}
%
并在第一行和第三行中标识方程 \eqRef{equ:chain_rule_cross_1} 。


\subsection{初等Jacobian块}
\label{sec:jacs_elementary}

\subsubsection{求逆(Inverse)}
\label{sec:Jac_inversion}

我们用方程 \eqRef{equ:Jacobian} 定义 
%
\begin{align}
\mjac{\cX\inv}{\cX} 
  &\te \rdpar{\cX\inv}{\cX} \Quad\in \bbR^{m\times m}
  ~. \\
%
\intertext{这可以通过使用方程 \eqRef{equ:prop_exp} 和方程 \eqRef{equ:Adj4} 的伴随来确定,}
%
\mjac{\cX\inv}{\cX}\label{equ:Jac_inv}
  &= \lim_{\bftau\to0}\frac{\Log((\cX\inv)\inv(\cX\Exp(\bftau))\inv)}{\bftau} \notag\\
  &= \lim_{\bftau\to0}\frac{\Log(\cX\Exp(-\bftau)\cX\inv)}{\bftau} \notag \\
  &= \lim_{\bftau\to0}\frac{(\cX(-\bftau)^\wedge\cX\inv)^\vee}{\bftau} 
  = -\Ad[\cM]{\cX}
~.
\end{align}
%


\subsubsection{组合(Composition)}
\label{sec:Jac_composition}

我们用方程 \eqRef{equ:Jacobian} 定义
%
\begin{align}
\mjac{\cX\circ\cY}{\cX} &\te \rdpar{\cX\circ\cY}{\cX} &&\in \bbR^{m\times m} \\
\mjac{\cX\circ\cY}{\cY} &\te \rdpar{\cX\circ\cY}{\cY} &&\in \bbR^{m\times m}
~,
  \intertext{并如上面一样使用方程 \eqRef{equ:Adj4} 和方程 \eqRef{equ:Adj5},}
\mjac{\cX\circ\cY}{\cX} &= \Ad[\cM]{\cY}\inv \label{equ:Jac_comp_1}\\
\mjac{\cX\circ\cY}{\cY} &= \bfI \label{equ:Jac_comp_2}
\end{align}


\subsubsection[Right and left Jacobians]{Jacobians of $\cM$}

我们定义流形 $\cM$ 的右Jacobian矩阵(\emph{right Jacobian of $\cM$}) 为 $\cX=\Exp(\bftau)$ 的右Jacobian矩阵,即,对于 $\bftau\in
\bbR^m$ ,
%
\begin{align}
\mjac{}{r}(\bftau) \te \rdpar{\Exp(\bftau)}{\bftau} \in \bbR^{m\times m} 
~,
\label{equ:M_Jr}
\end{align}
%
这由方程 \eqRef{equ:Jacobian} 定义。 
右Jacobian矩阵将参数 $\bftau$ 的变化映射到 $\Exp(\bftau)$处的局部(\emph{local})切空间中的变化。
从方程 \eqRef{equ:Jacobian} 这很容易证明,对于小的 $\delta\bftau$ 值,以下近似值成立,
%
\begin{align}
\Exp(\bftau+\delta\bftau) &\approx \Exp(\bftau)\Exp(\mjac{}{r}(\bftau)\delta\bftau) \label{equ:Jr_1} \\
\Exp(\bftau)\Exp(\delta\bftau) &\approx \Exp(\bftau+\mjac{-1}{r}(\bftau)\,\delta\bftau) \label{equ:Jr_2} \\
\Log(\Exp(\bftau)\Exp(\delta\bftau)) &\approx \bftau+\mjac{-1}{r}(\bftau)\,\delta\bftau  \label{equ:Jr_3}
~.
\end{align}
%

作为补充,流形 $\cM$ 的左Jacobian矩阵(\emph{left Jacobian of $\cM$})被定义为,
%
\begin{align}
\mjac{}{l}(\bftau) \te \ldpar{\Exp(\bftau)}{\bftau} \in \bbR^{m\times m} 
~,
\label{equ:M_Jl}
\end{align}
%
使用左Jacobian矩阵方程 \eqRef{equ:left-Jacobian},得出近似值
%
\begin{align}
\Exp(\bftau+\delta\bftau) &\approx \Exp(\mjac{}{l}(\bftau)\delta\bftau)\Exp(\bftau)  \label{equ:Jl_1}\\
\Exp(\delta\bftau)\Exp(\bftau) &\approx \Exp(\bftau+\mjac{-1}{l}(\bftau)\,\delta\bftau)  \label{equ:Jl_2}\\
\Log(\Exp(\delta\bftau)\Exp(\bftau)) &\approx \bftau+\mjac{-1}{l}(\bftau)\,\delta\bftau  \label{equ:Jl_3}
~.
\end{align}
% 
左Jacobian矩阵将参数 $\bftau$ 的变化映射到全局(\emph{global})切空间或Lie代数中的变化。
从方程 ~\eqssRef{equ:Jr_1,equ:Jl_1} 我们可以把左Jacobian矩阵和右Jacobian矩阵用伴随联系起来,
%
\begin{align}\label{equ:Jr_Jl_Adj}
\Ad[\cM]{\Exp(\bftau)} = \mjac{}{l}(\bftau)\,\mjac{}{r}\inv(\bftau)
~.
\end{align}
%
此外,链式法则允许我们关联 $\mjac{}{r}$ 和 $\mjac{}{l}$,
%
\begin{align}\label{equ:Jr_minus}
\mjac{}{r}(-\bftau) 
  &\te \mjac{\Exp(-\bftau)}{-\bftau} 
  = \mjac{\Exp(-\bftau)}{\bftau}\mjac{\bftau}{-\bftau} 
  = \mjac{\Exp(\bftau)\inv}{\bftau}(-\bfI) 
  \notag\\
  &
  = -\mjac{\Exp(\bftau)\inv}{\Exp(\bftau)}\mjac{\Exp(\bftau)}{\bftau} 
  = \Ad{\Exp(\bftau)}\mjac{}{r}(\bftau)
  \notag\\
  &
  = \mjac{}{l}(\bftau)
  ~.
\end{align}



对于使用中的经典流形, $\mjac{}{r}$、 $\mjac{}{r}\inv$、 $\mjac{}{l}$ 和 $\mjac{}{l}\inv$ ,存在封闭形式。
查询请参考附件。


\subsubsection{群作用}


对于 $\cX\in\cM$ 和 $v\in\cV$我们用方程 \eqRef{equ:Jacobian} 定义
%
\begin{align}
\mjac{\cX\cdot v}{\cX} &\te \rdpar{\cX\cdot v}{\cX} \\
\mjac{\cX\cdot v}{v}   &\te \rdpar{\cX\cdot v}{v} 
~.
\end{align}
%
由于群作用依赖于集合 $\cV$,因此这些表达式不能通用化。
查询请参考附件。



\subsection{有用的,但可推导的, Jacobian 矩阵块}

\subsubsection[Log map]{$\Log$ 映射}

对于 $\bftau=\Log(\cX)$ ,
并从方程 \eqRef{equ:Jr_3} ,
%
\begin{align}\label{equ:Jac_log}
\mjac{\Log(\cX)}{\cX} 
&= \mjac{-1}{r}(\bftau) 
~.
\end{align}

\subsubsection{加号和减号}

我们有
%
\begin{align}
\mjac{\cX\op\bftau}{\cX}
  &= \mjac{\cX\circ(\Exp(\bftau))}{\cX} 
  ~~~~~~~~~= \Ad[\cM]{\Exp(\bftau)}\inv \\
\mjac{\cX\op\bftau}{\bftau}
  &= \mjac{\cX\circ(\Exp(\bftau))}{\Exp(\bftau)}\mjac{\Exp(\bftau)}{\bftau}
  = \mjac{}{r}(\bftau) 
\end{align}
%
并得到 $\cZ=\cX\inv\circ\cY$ 和 $\bftau=\cY\om\cX=\Log(\cZ)$ ,
%
\begin{align}
\mjac{\cY\om\cX}{\cX}
  &= \mjac{\Log(\cZ)}{\cZ}\mjac{\cZ}{\cX\inv}\mjac{\cX\inv}{\cX} 
   = -\mjac{-1}{l}(\bftau)
  \\
\mjac{\cY\om\cX}{\cY}
  &= \mjac{\Log(\cZ)}{\cZ}\mjac{\cZ}{\cY} 
  ~~~~~~~~~= \mjac{-1}{r}(\bftau)
~.
\end{align}
%
在这里证明前者
%
\begin{align*}
\mjac{\cY\om\cX}{\cX}
  &= \mjac{\Log(\cX\inv\circ\cY)}{(\cX\inv\circ\cY)}\,\mjac{(\cX\inv\circ\cY)}{\cX\inv}\,\mjac{\cX\inv}{\cX} 
 \notag\\ 
 (\ref{equ:Jac_log},\ref{equ:Jac_comp_1},\ref{equ:Jac_inv})
  &= ~~\,\mjac{-1}{r}(\bftau)~~\Ad[\cM]{\cY}\inv~~(-\Ad[\cM]{\cX}) 
 \notag\\ 
 (\ref{equ:Adj5},\ref{equ:Adj7})
  &= -\mjac{-1}{r}(\bftau)\,\Ad[\cM]{\cY\inv\cX} 
 \notag\\ 
  &= -\mjac{-1}{r}(\bftau)\,\Ad[\cM]{\Exp(\bftau)}\inv
  \notag\\
 (\ref{equ:Jr_Jl_Adj})
  &
  =  -\mjac{-1}{l}(\bftau)
  ~.
\end{align*}




