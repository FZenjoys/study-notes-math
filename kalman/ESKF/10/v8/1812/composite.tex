% !TEX root = micro_Lie_theory.tex


\section{组合流形}

以失去与Lie理论的某些一致性为代价,但以获得符号和操作方面的一些优势为好处,人们可以将大的和非均匀的状态视为流形组合(或捆绑)。


\if \examples y % !TEX root = micro_Lie_theory.tex

%%%%%%%%%%%%%%%%%%%%%%%% Composite %%%%%%%%%%%%%%%%%%%%%%%
\begin{fexample}{$\SE(n)$ \emph{vs.} $T(n) \tcross \SO(n)$ \emph{vs.} 
$\langle\bbR^n,\SO(n)\rangle$}
%the composite}
\label{ex:sen_sonxrn_comp}

我们考虑平移空间 $\bft\in\bbR^n$ 和旋转空间 $\bfR\in\SO(n)$。
为此,我们有著名的 $\SE(n)$ 刚体运动流形 $\bfM=\begin{bsmallmatrix}\bfR&\bft\\\bf0&1\end{bsmallmatrix}$ (参见 Apps.~\ref{sec:SE2} 和 \ref{sec:SE3}),这也可以构造为 $T(n) \tcross \SO(n)$ (参见 Apps.~\ref{sec:S1_SO2}, \ref{sec:S3_SO3} 和 \ref{sec:Tn})。
这两者非常相似,但有不同的正切参数化:
当 $\SE(n)$ 使用 $\bftau=(\bth,\bfrho)$ 和 $\bfM=\exp(\bftau^\wedge)$ , 
而 $T(n) \tcross \SO(n)$ 使用 $\bftau=(\bth,\bfp)$ 和 $\bfM=\exp(\bfp^\wedge)\exp(\bth^\wedge)$ 。
它们共享旋转部分 $\bth$,但显然是 $\bfrho\ne\bfp$ (对于更多详细信息,参见文献 \cite[pag.~35]{CHIRIKJIAN-11})。
简而言之,$\SE(n)$ 作为一个连续体同时执行平移和旋转,
而 $T(n) \tcross \SO(n)$ 执行链式平移+旋转。
相反,在组合 $\langle\bbR^n,\SO(n)\rangle$ 中,旋转和平移根本不相互作用。
通过用 $\Exp()$ 将组合结合,我们得到(右)加号算子,
%
\begin{align*}
\SE(n)&:& 
\bfM\oplus\bftau &= \begin{bmatrix}
\bfR\Exp(\bth) & \bft+\bfR\bfV(\bth)\bfrho \\
\bf0 & 1
\end{bmatrix} 
\\
T(n) \tcross \SO(n)&:&
\bfM\oplus\bftau &= \begin{bmatrix}
\bfR\Exp(\bth) & \bft+\bfR\bfp \\
\bf0 & 1
\end{bmatrix} 
\\ 
\langle\bbR^n,\SO(n)\rangle&:&
\bfM\dplus\bftau 
& 
= 
\begin{bmatrix}
\bft+\bfp \\
\bfR\Exp(\bth)
\end{bmatrix} 
\end{align*}
%
其中 $\oplus$ 可用于系统动力学,例如运动积分,但通常不用 $\dplus$,后者可用于对扰动进行建模。
%
它们各自的减号算子读取,
%. 
\begin{align*}
\SE(n)&:&
\bfM_2\ominus\bfM_1 & = \begin{bmatrix}
\bfV_1\inv\bfR_1\tr(\bfp_2 - \bfp_1) \\ \Log(\bfR_1\tr\bfR_2)
\end{bmatrix} 
\\
T(n) \tcross \SO(n)&:&
\bfM_2\ominus\bfM_1 & = \begin{bmatrix}
\bfR_1\tr(\bfp_2 - \bfp_1) \\ \Log(\bfR_1\tr\bfR_2)
\end{bmatrix} 
\\
\langle\bbR^n,\SO(n)\rangle&:&
\bfM_2\dminus\bfM_1 
&= 
\begin{bmatrix}
\bfp_2 - \bfp_1 \\ \Log(\bfR_1\tr\bfR_2)
\end{bmatrix} 
~,
\end{align*}
%
现在这里有趣的是, $\dminus$ 可以用来评估误差和不确定性。这使得 $\dplus,\dminus$ 成为计算导数和协方差的有价值的算子。
\end{fexample}
%%%%%%%%%%%%%%%%%%%%%%%%%%%%%%%%%%%%%%%%%%%%%%%%%%
 \fi


一个组合流形(\emph{composite manifold}) $\cM=\langle\cM_1,\cdots,\cM_M\rangle$ 不小于 $M$ 个非相互作用的流形的级联。
这源于幺元、逆元的定义,以及单独地在每一个块上的组合作用的组合,
%
\begin{align}
\cE_\diamond &\te \begin{bmatrix}
\cE_1 \\ \vdots \\ \cE_M
\end{bmatrix},
&
\cX^\diamond &\te \begin{bmatrix}
\cX\inv \\ \vdots \\ \cX_M\inv
\end{bmatrix},
&
\cX\diamond\cY &\te \begin{bmatrix}
\cX\circ\cY_1 \\
\vdots\\
\cX_M\circ\cY_M 
\end{bmatrix}
,
\end{align}
%
从而实现了群公理,以及一个非相互作用的收回映射,为了统一符号(注意尖括号),我们还将其标记为“指数映射”,
%
\begin{align}\label{equ:exp_composite}
\Exp\langle\bftau\rangle &\te \begin{bmatrix}
\Exp(\bftau_1) \\ \vdots \\ \Exp(\bftau_M)
\end{bmatrix}
\,,
&
\Log\langle\cX\rangle &\te \begin{bmatrix}
\Log(\cX) \\ \vdots \\ \Log(\cX_M)
\end{bmatrix}
,
\end{align}
% 
从而确保平滑。
这些产生了组合的右结合(right-)的加号和减号(注意菱形符号),
%
\begin{align}
\cX\dplus\bftau &\te \cX\diamond\Exp\langle\bftau\rangle \\
\cY\dminus\cX &\te \Log\langle\cX^\diamond\diamond\cY\rangle
~.
\end{align}

这些考虑的关键结果%
\if\examples y{ (参见 Ex.~\ref{ex:sen_sonxrn_comp}) }\else { }\fi 
是可以定义新的导数,\footnotemark\ 使用 $\dplus$ 和 $\dminus$ ,
\footnotetext{这里我们假设右导数,但同样适用于左导数。}
%
\begin{align}
\ndpar{f(\cX)}{\cX} \te \lim_{\bftau\to0}\frac{f(\cX\dplus\bftau)\dminus f(\cX)}{\bftau}
\label{equ:Jacobian_composite}
~.
\end{align}
%
利用这个导数,作用在组合流形上的函数 $f:\cM\to\cN$ 的Jacobian矩阵可以按块计算来确定,这就产生了只需要知道组合的流形块的简单表达式,
%
\begin{align}
\ndpar{f(\cX)}{\cX} &= \begin{bmatrix}
\ndpar{f_1}{\cX_1} & \cdots & \ndpar{f_1}{\cX_M} \\
\vdots             & \ddots & \vdots \\
\ndpar{f_N}{\cX_1} & \cdots & \ndpar{f_N}{\cX_M} \\
\end{bmatrix}
~,
\end{align}
%
其中, $\ndpar{f_i}{\cX_j}$ 分别用方程 \eqRef{equ:Jacobian} 计算。 
对于 $\bftau$ 的小值,以下方程成立,
%
\begin{align}\label{equ:lin_approx_composite}
f(\cX\dplus\bftau) \xrightarrow[{\bftau}\to0]{} f(\cX)\dplus\ndpar{f(\cX)}{\cX}\,{\bftau}
\quad \in \cN
~.
\end{align}

当使用这些导数时,协方差和不确定性传播必须遵循约定。特别是,协方差矩阵方程 \eqRef{equ:cov} 变成
%
\begin{align}\label{equ:cov_composite}
\bfSigma_\cX \te \bbE[(\cX \dminus \bar\cX)(\cX \dminus \bar\cX)\tr]~\in\bbR^{n\times n}
~,
\end{align}
%
对于线性化传播方程 \eqRef{equ:cov_propagation} ,使用方程 \eqRef{equ:Jacobian_composite} 以应用它。

