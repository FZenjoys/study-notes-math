% !TEX root = micro_Lie_theory.tex

%%%%%%%%%%%%%%%%%%%%%%%%%%%%%  S3  %%%%%%%%%%%%%%%%%%%%%%%%%%%%%%%%
\begin{fexample}	{单位四元数群 $S^3$}
\label{ex:S3_intro}
Lie群的第二个例子,也相对容易可视化,是四元数乘法下的单位四元数群 (\figRef{fig:manifold_q})。
单位四元数的形式 $\bfq=\cos(\theta/2)+\bfu\sin(\theta/2)$,其中 $\bfu=iu_x+ju_y+ku_z$ 是一个单位旋转轴,并且 $\theta$ 是一个旋转角度。

\emph{-- 作用:}
向量 $\bfx=ix+jy+kz$ 在三维空间中通过双四元数乘积 $\bfx'=\bfq\,\bfx\,\bfq^*$ 围绕单位旋转轴 $\bfu$ 旋转一个角度 $\theta$ 。

\emph{-- 群事实:} 
单位四元数的乘积是单位四元数,幺元为 $1$,并且逆元是共轭 $\bfq^*$。

\emph{-- 流形事实:} 
单位范数约束定义了三维球面 $S^3$,一个球形的三维曲面或4维空间中的流形(\emph{manifold})。
在这个曲面上,单位四元数随时间而演变。
该群(该球面)在局部重新组装线性空间(切超平面 $\bbR^3\subset\bbR^4$),而不是在全局重新组装。
\end{fexample}
%%%%%%%%%%%%%%%%%%%%%%%%%%%%%%%%%%%%%%%%%%%%%%%%%%%%%%%%%%%%%
