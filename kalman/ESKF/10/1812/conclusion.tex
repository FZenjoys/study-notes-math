% !TEX root = micro_Lie_theory.tex

\section{结论}

我们以一种对精通状态估计的听众有用的形式介绍了Lie理论的本质,重点是机器人技术的应用。
我们通过以下几项举措做到了这一点:

第一,选择尽可能避免抽象数学概念的素材。这有助于聚焦Lie理论,使其工具更容易理解和使用。

第二,我们选择了一种教学方法,有明显的冗余。正文是一般性的,涵盖了Lie理论的抽象观点。它陪伴有打包的例子,其中基础的抽象概念,以特定的Lie群,以及大量的插图和非常冗长的标题。

第三,我们推广了便利算子的使用,例如大写的 $\Exp()$ 和 $\Log()$ 映射,以及加号和减号算子 $\op,\,\om, \dplus, \dminus$。它们允许我们研究切空间的笛卡尔表示,产生导数和协方差处理的公式,这些公式与标准向量空间中的相应公式非常相似。

第四,我们特别强调了Jacobian矩阵的定义、几何解释和计算。为此,我们引入了Jacobian矩阵和协方差的符号,允许进行视觉上强大的操作。特别是,链式规则用这种符号清晰可见。这有助于建立直觉并减少错误。

第五,我们在附录中介绍了机器人学中最常见群的广泛公式概要。在二维空间中,我们给出了单位复数 $S^1$ 和旋转矩阵 $\SO(2)$ 的旋转群和刚体运动群 $\SE(2)$。在三维空间中,我们给出了用于旋转的单位四元数群 $S^3$ 和旋转矩阵 $\SO(3)$,以及刚体运动群 $\SE(3)$。我们还给出了任意维度的平移群,它们可以由标准向量空间 $\bbR^n$ 在加法下实现,也可以由矩阵平移群 $T(n)$ 在乘法下实现。

第六,通过一些应用实例说明了Lie理论在解决机器人问题上的优雅性和精确性。
组合群这个有点幼稚的概念有助于将异构状态向量统一成一种Lie理论形式。 

最后,陪伴着正文我们使用新的C++库 \manif\ \cite{DERAY-20-manif} 来实现这里所描述的工具。 \manif\ 可以在 \url{https://github.com/artivis/manif} 这里被找到。
在 \secRef{sec:SLAM} 中的应用程序以 \manif\ 为例进行演示。

虽然我们没有介绍任何新的理论素材,但我们相信,Lie理论在这里所呈现的形式将有助于许多研究人员进入该领域,为他们的未来发展提供帮助。
我们还认为,仅这一点就代表了一种宝贵的贡献。
