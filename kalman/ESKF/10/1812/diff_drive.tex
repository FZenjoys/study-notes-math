% !TEX root = micro_Lie_theory.tex

\section{差动驱动自校正}
\label{sec:diff_drive}

%%%%%%%%%%%%%%%%%%%%%%%%%%%%%%%%%%%%%%%%%%%%%%%%%%%%%%%%%%%%
\subsection{概述}

差动驱动模型由两种驱动轮组成,一种在单轴上,一种位于机器人基座的每一侧,
其原点坐标系位于轴线的中心。
机器人的参数化由
它的轮子的半径 $(r_l, r_r)$ 和轴线的长度 $d$ 确定。
与每一个参数中的相关联的是一个校正因子,使得
$\bfc=[c_l~c_r~c_d]\tr$ 是内在参数校正向量。
%
通常通过车轮编码器测量运动,报告车轮增量角度 $\bfy=[\delta\psi_l, \delta\psi_r]+\bfn$ ,
在每个时间步长 $\delta t$ 处,其中 $\bfn$ 是加性高斯噪声。


%%%%%%%%%%%%%%%%%%%%%%%%%%%%%%%%%%%%%%%%%%%%%%%%%%%%%%%%%%%%
\subsection{State and delta definitions}
%
We define the states of position and orientation angle
$\bfx = (\bfp,\theta)$, which act as a compact representation of $SE(2)$.
Consequently, state increments or deltas $\D_{ij}$ and $\delta_k$ are also in $SE(2)$.


%
%%%%%%%%%%%%%%%%%%%%%%%%%%%%%%%%%%%%%%%%%%%%%%%%%%%%%%%%%%%%
\subsection{Delta pre-integration}

For convenience, we define the body magnitudes 
$\bfb = (\delta l,\delta \theta)=f_b(\bfy,\bfc,\bfn)$ as
%
%
\begin{align}
\label{equ:diff_drive_psis}
\begin{split}
\delta l &= \tfrac12(r_r c_r \delta\psi_r + r_l c_l\delta\psi_l)  \\
\delta \theta &= \tfrac{1}{D}(r_r c_r \delta\psi_r - r_l c_l\delta\psi_l) 
~.
\end{split}
\end{align}
%
with $D=d\,c_d$. 
Its components, respectively, highlight the common (along track length), and differential (turn) components of the wheels reported motions.

Assuming constant control inputs over the sampling time period $[t_j,t_k]$, the robot
moves along an arc of circle of radius
%
\begin{align}
R = \frac{\delta l}{\delta \theta} 
~.
\end{align}
%
The motion increment $\delta_k$ can be expressed in $SE(2)$ as
%
\begin{align}
\label{equ:diff_drive_exact_inte}
\delta x &= R\sin(\delta\theta) \notag\\
\delta y &= R(1-\cos(\delta\theta)) \\
\delta\theta &= \delta \theta \notag 
~,
\end{align}
%
which is exactly $\delta=\Exp(\tau)\in SE(2)$ with $\tau=[\delta l,0,\delta \theta]\in\se(2)$, \ie, assuming no lateral wheel slippage.
%
In case the robot follows a straight trajectory, we have $\delta \theta\to0$,
and $R\to+\infty$.
This case is handled by means of the approximation %using a mid-point integration:
%
\begin{align}
\label{equ:diff_drive_runge_kutta_inte}
%
%
\delta x &= \delta l \cos(\delta\theta/2) \notag \\
%
\delta y &= \delta l \sin(\delta\theta/2) \\
%
\delta\theta &= \delta \theta \notag 
~.
\end{align}


%%%%%%%%%%%%%%%%%%%%%%%%%%%%%%%%%%%%%%%%%%%%%%%%%%%%%%%%%%%%
%\subsection{Motion integration}

The delta integration is simply the composition in $SE(2)$,
%
\begin{align}\label{equ:diff_drive_composition}
\begin{split}
\Dp_{ik} &= \Dp_{ij} + \D\bfR_{ij}\delta \bfp_k \\
\D\theta_{ik} &= \D\theta_{ij} + \delta\theta_k
\end{split}
\end{align}
%
where $\D\bfR_{ij}=\Exp(\D\theta_{ij})$ is the rotation matrix delta corresponding to the
rotation angle of $\D\theta_{ij}$, see \eqRef{equ:R_SO2}, and $\delta \bfp_k=(\delta x,\delta y)$ is the translation vector delta of $\delta_k$.


%%%%%%%%%%%%%%%%%%%%%%%%%%%%%%%%%%%%%%%%%%%%%%%%%%%%%%%%%%%%
\subsection{Jacobians}
\label{sec:diff_drive_jacs}
%
\subsubsection{Jacobians of the motion components}
%
From \eqRef{equ:diff_drive_psis},
%
\begin{subequations}
\label{equ:diff_drive_psis_jac}
\begin{align}
\jac{\bfb}{\bfy} = \jac{\bfb}{\bfn} &= \begin{bmatrix}
\frac12 r_l c_l & \frac12 r_r c_r  \\
- r_l c_l       & r_r c_r
\end{bmatrix}
&&\in \bbR^{2\tcross2}
\\
\jac{\bfb}{\bfc} &= \begin{bmatrix}
\frac12 \psi_{l} r_l & \frac12 \psi_{r} r_r & 0\\
-\psi_{l} r_l        & \psi_{r} r_r & -\frac{\delta\theta}{c_d}
\end{bmatrix}
&&\in \bbR^{2\tcross3}
\end{align}
\end{subequations}



\subsubsection{Jacobians of the current delta}

From \eqRef{equ:diff_drive_exact_inte},
%
\begin{align}\label{equ:diff_drive_exact_inte_jac}
\jac{\delta_k}{\bfb} &= 
\begin{bmatrix}
\frac{\sin(\delta\theta)}{\delta\theta} 
&
R\big(\cos(\delta\theta)-\frac{\sin(\delta\theta)}{\delta\theta}\big)\\
\frac{1-cos(\delta\theta)}{\delta\theta} 
&
R\big(\sin(\delta\theta)-\frac{1-\cos(\delta\theta)}{\delta\theta}\big)\\
0 & 1
\end{bmatrix}
\in \bbR^{3\tcross2}
\end{align}
%
Similarly, from \eqRef{equ:diff_drive_runge_kutta_inte},
%
\begin{align}\label{equ:diff_drive_runge_kutta_meas_jac}
\jac{\delta_k}{\bfb} &=
\begin{bmatrix}
\cos(\delta\theta/2) &-\frac12\delta l\sin(\delta\theta/2)\\
\sin(\delta\theta/2) &\phantom{-}\frac12\delta l\cos(\delta\theta/2)\\
0 & 1
\end{bmatrix}
\in \bbR^{3\tcross2}
\end{align}
%
\subsubsection{Jacobians of the delta composition}
%
From \eqRef{equ:diff_drive_composition}, % is in $SE(2)$
%
\begin{subequations}\label{equ:jac_composition_DD}
\begin{align}
\jac{\D_{ik}}{\D_{ij}} &= \begin{bmatrix}
\bfI & \DR_{ij}\hatx{1}\delta\bfp_k \\
\bf0 & 1
\end{bmatrix}
&&\in \bbR^{3\tcross3}
\\
\jac{\D_{ik}}{\delta_k} &= \begin{bmatrix}
\DR_{ij} & \bf0 \\
\bf0        & 1
\end{bmatrix}
&&\in \bbR^{3\tcross3}
\end{align}
\end{subequations}
%
with $\hatx{1}=\begin{bsmallmatrix}0 & -1 \\ 1 & 0
\end{bsmallmatrix}$, see \appRef{sec:derivatives_SO2}.
%
%where $\DR_{ij}$ is the rotation matrix delta corresponding to the rotation angle of $\D_{ij}$
%and $\delta \bfp_k$ is the translation vector delta corresponding to $\delta_k$.

%%%%%%%%%%%%%%%%%%%%%%%%%%%%%%%%%%%%%%%%%%%%%%%%%%%%%%%%%%%%
\subsection{Integration of the delta covariance and the Jacobian}

We follow the general procedure, obtaining $\bfQ_k$ from \eqRef{equ:Q_d}.
Here, $\bfQ_s$ is a small noise diagonal covariance accounting for wheel slippage, and $\bfQ_n$ is defined by, 
%
\begin{align}
\label{equ:diff_drive_meas_cov}
\bfQ_n &=
\begin{bmatrix}
\sigma^2_{\psi_l}+\alpha^2 & 0\\
0 & \sigma^2_{\psi_r}+\alpha^2
\end{bmatrix}, \\
\sigma_{\psi_l}^2&=k_l \delta\psi_l,~~\sigma_{\psi_r}^2=k_r \delta\psi_r,~~\alpha=\tfrac12(\mu_l+\mu_r)  \notag
~,
\end{align}
%
where $k_r$ and $k_l$ are constant parameters,
and $\alpha$ acts as an offset equal to half the wheels encoders resolution $\mu_l$ and $\mu_r$
\cite{siegwart2011introduction}.
The pre-integrated delta covariance $\bfQ_{ik}$ is then integrated normally with \eqRef{equ:Q_D}, starting at $\bfQ_{ii}={\bf0}_{3\tcross3}$.

For the Jacobian we use  \eqRef{equ:jac_integration}, starting at $\jac{\D_{ii}}{\bfc} = {\bf0}_{3\tcross 3}$.


%%%%%%%%%%%%%%%%%%%%%%%%%%%%%%%%%%%%%%%%%%%%%%%%%%%%%%%%%%%%
\subsection{Residual}

Following \algRef{alg:residual_lupton} with $\dminus$ for errors, we have
%
\begin{align}\label{equ:diff_drive_residual}
\widehat\D_{ij} &= \begin{bmatrix}
\bfR_{i}\tr(\bfp_j-\bfp_i) \\ \theta_j-\theta_i 
\end{bmatrix} \\
\D_{ij} &= \D_{ij} \oplus \jac{\D_{ij}}{\bfc}(\bfc_i-\ol\bfc_i) \\
\bfr(\bfx_i,\bfx_j,\bfc_{i})
&= \bfOmega_{ij}^{\top/2} \begin{bmatrix}
\widehat{\Dp}_{ij}-\Dp_{ij} \\
\widehat{\D\theta}_{ij} - \D\theta_{ij}
\end{bmatrix}
\in \bbR^3
~,
\end{align}
%
where the angle differences must be brought to $(-\pi,\pi\,]$.
The information matrix is given by $\bfOmega_{ij}=\bfQ_{\D ij}\inv$.



