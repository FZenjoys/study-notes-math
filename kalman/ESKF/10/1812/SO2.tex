% !TEX root = micro_Lie_theory.tex


\section{二维旋转群 $S^1$ 和 $SO(2)$}
\label{sec:S1_SO2}

Lie群的 $S^1$ 是复数积下的单位复数群。
它的拓扑是单位圆,或者单位一维球面,因此名为 $S^1$。
群,Lie代数和向量元素的形式是,
%
\begin{align}
\bfz&=\cos\theta+i\sin\theta, & \tau^\wedge&=i\theta, & \tau&=\theta
~.
\end{align}
%
求逆和组合是通过共轭 $\bfz\inv = \bfz^*$ 和乘积 $\bfz_a\circ\bfz_b = \bfz_a\,\bfz_b$ 达成的。

群 $\SO(2)$ 是在矩阵乘法下平面上的特殊正交矩阵或旋转矩阵的群。
群,Lie代数和向量元素的形式,
%
\begin{align}
\bfR&= \begin{bsmallmatrix}
 \cos\theta & -\sin\theta \\ \sin\theta & \cos\theta 
 \end{bsmallmatrix}
, & \tau^\wedge&=\hatx{\theta}\te \begin{bsmallmatrix}
0 & -\theta \\ \theta & 0
\end{bsmallmatrix}, & \tau&=\theta
~.
\end{align}
%
求逆和组合是通过共轭 $\bfR\inv = \bfR\tr$ 和乘积 $\bfR_a\circ\bfR_b = \bfR_a\,\bfR_b$ 达成的。

两个群都旋转$2$元向量,并且它们具有同构的切空间。
因此,我们一起研究它们。

\subsection{Exp 和 Log 映射}

Exp 和 Log 映射可以定义为复数 $S^1$ 和旋转矩阵 $SO(2)$ 。
对于 $S^1$ 我们有,
%
\begin{align}
\bfz = \Exp(\theta) &= \cos\theta+i\sin\theta && \in\bbC \label{equ:Euler_formula}\\
\theta = \Log(\bfz) &= \arctan(\Im(\bfz),\Re(\bfz)) && \in\bbR
~,
%
\intertext{其中方程 \eqRef{equ:Euler_formula} 是欧拉公式, 而对于 $SO(2)$ ,}
%
\bfR = \Exp(\theta) &= \begin{bmatrix}
\cos\theta & -\sin\theta \\ \sin\theta & \cos\theta
\end{bmatrix} &&\in\bbR^{2\tcross2} \label{equ:R_SO2} \\
\theta = \Log(\bfR) &= \arctan(r_{21},r_{11}) && \in\bbR
~.
\end{align}
%


\subsection{求逆、组合、指数映射}

我们考虑通用的二维旋转元素,并用无衬线字体 $\sQ,\sR$ 来标记它们。
我们有
%
\begin{align}
\sR(\theta)\inv &= \sR(-\theta) \\
\sQ\circ\sR   &= \sR\circ\sQ 
~,
%
\intertext{即,平面旋转是可交换的。
因此,}
%
\Exp(\theta_1+\theta_2) &= \Exp(\theta_1)\circ\Exp(\theta_2) \\
\Log(\sQ\circ\sR) &= \Log(\sQ)+\Log(\sR) \\
\sQ\om\sR &= \theta_Q-\theta_R 
~.
\end{align}

\subsection{Jacobian矩阵块}
\label{sec:derivatives_SO2}

由于我们定义的导数映射切向量空间,并且这些空间重叠于 $S^1$ 和 $SO(2)$ 的平面旋转流形,即,$\theta=\Log(\bfz)=\Log(\bfR)$,因此Jacobian矩阵独立于所使用的表示($\bfz$ 或 $\bfR$)。

\subsubsection[Adjoint and other Jacobians]{伴随与其它平凡Jacobian矩阵}\label{sec:SO2_jacs}
%
从方程 \eqRef{equ:Jacobian}, \secRef{sec:jacs_elementary} 和上述性质,下面的标量导数块变得平凡,
%
\begin{align}
\Ad[\SO(2)]{\sR} &= 1 && \in\bbR \\
\mjac{\sR\inv}{\sR} 
 &= -1 && \in \bbR\\
\mjac{\sQ\circ\sR}{\sQ} 
 = \mjac{\sQ\circ\sR}{\sR} 
 &= 1 && \in \bbR\\
\bfJ_r(\theta)
 = \bfJ_l(\theta)
 &= 1 && \in \bbR \\
\mjac{\sR\op\theta}{\sR}   
 =~~\,\mjac{\sR\op\theta}{\theta}   
 & =1 && \in \bbR\\
\mjac{\sQ\om\sR}{\sQ} 
 = -\mjac{\sQ\om\sR}{\sR} 
 &= 1 && \in \bbR
\end{align}
%
%%%%%%%%%%%%%%%%%%%%%


\subsubsection{旋转作用}
\label{sec:jac_SO2_action}

对于作用 $\sR\cdot\bfv$ 我们有,
%
\begin{align}
\mjac{\sR\cdot\bfv}{\sR}
&= \lim_{\theta\to0}\frac{\bfR\Exp(\theta)\bfv-\bfR\bfv}{\theta} \notag \\
&= \lim_{\theta\to0}\frac{\bfR(\bfI+\hatx{\theta})\bfv-\bfR\bfv}{\theta} \notag \\
&= \lim_{\theta\to0}\frac{\bfR\hatx{\theta}\bfv}{\theta} 
 = \bfR\hatx{1}\bfv && \in \bbR^{2\times 1} 
%
\intertext{并且}
%
\mjac{\sR\cdot\bfv}{\bfv} &= \ndpar{\bfR\bfv}{\bfv} = \bfR && \in\bbR^{2\times2}
~.
\end{align}
%

