% !TEX root = micro_Lie_theory.tex

\section{平移群 $(\bbR^n,+)$ 和 $T(n)$}
\label{sec:Tn}

群 $(\bbR^n,+)$ 是加法下的向量群,并且可以被看作是一个平移群。我们认为它是平凡的(\emph{trivial}),因为群元素、Lie代数和切空间都是一样的,所以 $\bft=\bft^\wedge=\Exp(\bft)$。
%Put otherwise: the group and its tangent space are the same space.
它的等价矩阵群(在乘法下)是平移群 $T(n)$,它的群、Lie代数和切向量元素是,
%
\begin{align*}
\bfT &\te \begin{bmatrix}
\bfI & \bft \\ \bf0 & 1
\end{bmatrix} \in T(n),
&
\bft^\wedge &\te \begin{bmatrix}
\bf0 & \bft \\ \bf0 & 0
\end{bmatrix}\in\frak{t}(n),
&
\bft &\in\bbR^n
~.
\end{align*}
%
等价性很容易通过观察 $\bfT(\bf0)=\bfI$, $\bfT(-\bft)=\bfT(\bft)\inv$,以及交换组合来验证。
%
\begin{align*}
\bfT_1\bfT_2 = \begin{bmatrix}
\bfI & \bft_1+\bft_2 \\ \bf0 & 1
\end{bmatrix}
~,
\end{align*}
%
有效地将向量 $\bft_1$ 和 $\bft_2$ 一起相加。
由于在 $\bbR^n$ 中的总和是可交换的,所以在 $T(n)$ 中的组合乘积也是如此。
%
由于 $T(n)$ 是 $\SE(n)$ 的一个子群,其中 $\bfR=\bfI$,我们可以通过使用 $\bfR=\bfI$ 取得方程 \eqssRef{equ:SE2_Exp,equ:SE3_Exp} 轻松地确定其指数映射,并推广到任意$n$,
%
\begin{align}
\Exp&:&\bbR^n\to T(n)~; &&
\bfT &= \Exp(\bft) 
 =\begin{bmatrix}
  \bfI & \bft \\ \bf0 & 1
 \end{bmatrix}
 ~.
%
\intertext{ $T(n)$ 指数也可以从 $\exp(\bft\hhat)$ 的泰勒展开式得到,注意这个 $(\bft\hhat)^2=\bf0$ 。
这直接证明了 $(\bbR^n,+)$ 群的等价指数,也就是特征,}
%
\Exp&:&\bbR^n\to \bbR^n &&
\bft &= \Exp(\bft)
~.
\end{align}
%
在 $\bbR^n$ 中这推导出平凡的、可交换的、与右结合(right-)和左结合(left-)类似的、加号和减号算子,
%
\begin{align}
\bft_1\op\bft_2   &= \bft_1+\bft_2 \\
\bft_2\ominus\bft_1 &= \bft_2-\bft_1
~.
\end{align}

\subsection{Jacobian矩阵块}

我们不清晰地表示 $T(n)$ 和 $\bbR^n$ 的平移,并注意它们 $\sS$ 和 $\sT$。
Jacobian矩阵是平凡的(与 \secRef{sec:SO2_jacs} 中的那些 $S^1$ 和 $\SO(2)$ 相比较),
%
\begin{align}
\Ad{\sT}       
 &= ~\bfI 
 &&\in\bbR^{n\times n} \\
\mjac{\sT\inv}{\sT} 
 &= -\bfI 
 &&\in\bbR^{n\times n} \\
\mjac{\sT\circ\sS}{\sT} = ~\mjac{\sT\circ\sS}{\sS} 
 &= \bfI 
 &&\in\bbR^{n\times n} \\
\mjac{}{r} = ~\mjac{}{l} ~~
 &= \bfI 
 &&\in\bbR^{n\times n} \\
\mjac{\sT\op\bfv}{\sT} = ~~\mjac{\sT\op\bfv}{\bfv} 
 &= \bfI 
 &&\in\bbR^{n\times n} \\
\mjac{\sS\om\sT}{\sS} = -\mjac{\sS\om\sT}{\sT} 
 &= \bfI 
 &&\in\bbR^{n\times n} 
~.
\end{align}

