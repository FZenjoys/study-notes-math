% !TEX root = micro_Lie_theory.tex


\section{三维刚体运动群 $SE(3)$}
\label{sec:SE3}


我们将三维刚体运动群 $\SE(3)$  的元素写为
%
\begin{align}
\bfM= \begin{bmatrix}
\bfR & \bft \\ \bf0 & 1
\end{bmatrix} \in \SE(3) \subset \bbR^{4\times4}
~,
\end{align}
%
其中 $\bfR\in\SO(3)$ 是一个旋转,并且 $\bft\in\bbR^3$ 是一个平移向量。
Lie 代数和向量的正切是由这些类型的元素形成的
%
\begin{align}
\bftau^\wedge
  &= \begin{bmatrix}\hatx{\bth} & \bfrho \\ \bf0 & 0\end{bmatrix} \in \se(3)
  ~~,
& 
\bftau
  &= \begin{bmatrix}\bfrho \\ \bth\end{bmatrix}\in\bbR^6 
~.
\end{align}
%


\subsection{求逆、组合}

求逆和组合分别用矩阵的求逆和乘积执行,
%
\begin{align}
\bfM\inv &= \begin{bmatrix}
\bfR\tr & -\bfR\tr\bft \\ \bf0 & 1
\end{bmatrix} 
\\
\bfM_a\,\bfM_b &= \begin{bmatrix}
\bfR_a\bfR_b & \bft_a+\bfR_a\bft_b \\ \bf0 & 1
\end{bmatrix} 
~.
\end{align}



\subsection{Exp 和 Log 映射}

Exp 和 Log 通过指数映射直接从向量的切空间 $\bbR^6\cong\se(3)=\mtan{\SE(3)}$ 实现 ---  有关推导,参见文献 \cite{EADE-Lie},
%
\begin{align}
  \bfM = \Exp(\bftau) 
    &\te \begin{bmatrix}\Exp(\bth) & \bfV(\bth)\,\bfrho \\ \bf0 & 1  \end{bmatrix} \label{equ:SE3_Exp} \\
  \bftau = \Log(\bfM) 
    &\te \begin{bmatrix} \bfV\inv(\bth)\, \bfp \\ \Log(\bfR) \end{bmatrix}~.
\end{align}
%
其中 (回想对于 $\Log(\bfM)$ 这个 $\bth=\theta\bfu=\Log(\bfR)$)
%
\begin{align}
\bfV(\bth) = \bfI 
  + \frac{1-\cos\theta}{\theta^2}\hatx{\bftheta}
  + \frac{\theta-\sin\theta}{\theta^3}\hatx{\bftheta}^2
~
\end{align}
%
其中,注意,这与方程 \eqRef{equ:SO3_Jl} 完全匹配。

\subsection{Jacobian矩阵块}
\label{sec:derivatives_SE3}

\subsubsection{伴随}

我们有 (参见 \exRef{ex:SE3_adjoint}),
%
\begin{align}
\Ad[\SE(3)]{\bfM}\bftau 
  &= (\bfM\bftau^\wedge\bfM\inv)^\vee \notag 
  = \begin{bmatrix}
  \bfR\bfrho+\hatx{\bft}\bfR\bth \\
  \bfR\bth
  \end{bmatrix} \notag 
  = \Ad[\SE(3)]{\bfM}\begin{bmatrix}
  \bfrho \\ \bth
  \end{bmatrix} \notag 
\end{align}
%
因此,
%
\begin{align}
  \Ad[\SE(3)]{\bfM} &= \begin{bmatrix}
  \bfR & \hatx{\bft}\bfR \\ 0 & \bfR
  \end{bmatrix} \in \bbR^{6\times6}
  ~.
\end{align}

\subsubsection{求逆、组合}

从 \secRef{sec:jacs_elementary} 我们有,
%
\begin{align}
\mjac{\bfM\inv}{\bfM} &= - \begin{bmatrix}
  \bfR & \hatx{\bft}\bfR \\ 0 & \bfR
  \end{bmatrix} \\
\mjac{\bfM_a\bfM_b}{\bfM_a} &=   \begin{bmatrix}
  \bfR_b\tr & -\bfR_b\tr\hatx{\bft_b} \\ 0 & \bfR_b\tr
  \end{bmatrix} \\
\mjac{\bfM_a\bfM_b}{\bfM_b} &= \bfI_6  
~.
\end{align}

\subsubsection{右Jacobian矩阵和左Jacobian矩阵}

左Jacobian矩阵的封闭形式及其逆式由 Barfoot 在文献 \cite{BARFOOT-14} 中给出,
%
\begin{subequations}
\begin{align}
\mjac{}{l}(\bfrho,\bth) &= \begin{bsmallmatrix}
\mjac{}{l}(\bth) & \bfQ(\bfrho,\bth) \\
\bf0 & \mjac{}{l}(\bth)
\end{bsmallmatrix}
\\
\mjac{-1}{l}(\bfrho,\bth) &= \begin{bsmallmatrix}
\mjac{-1}{l}(\bth) & -\mjac{-1}{l}(\bth)\,\bfQ(\bfrho,\bth)\,\mjac{-1}{l}(\bth) \\
\bf0 & \mjac{-1}{l}(\bth)
\end{bsmallmatrix}
\end{align}
\end{subequations}
%
其中 $\mjac{}{l}(\bth)$ 是 $\SO(3)$ 的左 Jacobian 矩阵, 参见方程 \eqRef{equ:SO3_Jl} ,并且
%
\newcommand{\rhox}{\bfrho_\times}
\newcommand{\bthx}{\bth_\times}
%
\begin{align}
\bfQ(\bfrho,\bth) =& 
  \frac12\rhox 
  + \frac{\theta\!-\!\sin\theta}{\theta^3}(\bthx\rhox+\rhox\bthx+\bthx\rhox\bthx) 
  \notag\\
  &- \frac{1\! - \!\frac{\theta^2}{2}\! -\! \cos\theta}{\theta^4}(\bthx^2\rhox+\rhox\bthx^2-3\bthx\rhox\bthx)
  \notag\\
  &-\frac12\left(\frac{1 -  \frac{\theta^2}{2} - \cos\theta}{\theta^4} 
                  - 3\frac{\theta-\sin\theta-\frac{\theta^3}{6}}{\theta^5}\right)
  \notag\\
  &\times (\bthx\rhox\bthx^2 + \bthx^2\rhox\bthx)
~.
\end{align}
%
右Jacobian矩阵及其逆矩阵使用方程 \eqRef{equ:Jr_minus} 获得,即, $\mjac{}{r}(\bfrho,\bth)=\mjac{}{l}(-\bfrho,-\bth)$ 。

\subsubsection{刚体运动作用}
\label{sec:jac_SE3_action}
%
在点 $\bfp$ 上我们有作用,
%
\begin{align}
\bfM\cdot\bfp &\te \bft+\bfR\bfp
~,
\end{align}
%
因此,从方程 \eqRef{equ:SE3_Exp} ,
%
\begin{align}
\mjac{\bfM\cdot\bfp}{\bfM} 
  &= 
  \lim_{\bftau\to0}\frac{\bfM\Exp(\bftau)\cdot\bfp - \bfM\cdot\bfp}{\bftau}
  = \begin{bmatrix}\bfR & -\bfR\hatx{\bfp} \end{bmatrix} \\
\mjac{\bfM\cdot\bfp}{\bfp} &= \bfR
~.
\end{align}
