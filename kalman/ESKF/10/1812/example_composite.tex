% !TEX root = micro_Lie_theory.tex

%%%%%%%%%%%%%%%%%%%%%%%% Composite %%%%%%%%%%%%%%%%%%%%%%%
\begin{fexample}{$\SE(n)$ \emph{vs.} $T(n) \tcross \SO(n)$ \emph{vs.} 
$\langle\bbR^n,\SO(n)\rangle$}
%the composite}
\label{ex:sen_sonxrn_comp}

我们考虑平移空间 $\bft\in\bbR^n$ 和旋转空间 $\bfR\in\SO(n)$。
为此,我们有著名的 $\SE(n)$ 刚体运动流形 $\bfM=\begin{bsmallmatrix}\bfR&\bft\\\bf0&1\end{bsmallmatrix}$ (参见 Apps.~\ref{sec:SE2} 和 \ref{sec:SE3}),这也可以构造为 $T(n) \tcross \SO(n)$ (参见 Apps.~\ref{sec:S1_SO2}, \ref{sec:S3_SO3} 和 \ref{sec:Tn})。
这两者非常相似,但有不同的正切参数化:
当 $\SE(n)$ 使用 $\bftau=(\bth,\bfrho)$ 和 $\bfM=\exp(\bftau^\wedge)$ , 
而 $T(n) \tcross \SO(n)$ 使用 $\bftau=(\bth,\bfp)$ 和 $\bfM=\exp(\bfp^\wedge)\exp(\bth^\wedge)$ 。
它们共享旋转部分 $\bth$,但显然是 $\bfrho\ne\bfp$ (对于更多详细信息,参见文献 \cite[pag.~35]{CHIRIKJIAN-11})。
简而言之,$\SE(n)$ 作为一个连续体同时执行平移和旋转,
而 $T(n) \tcross \SO(n)$ 执行链式平移+旋转。
相反,在组合 $\langle\bbR^n,\SO(n)\rangle$ 中,旋转和平移根本不相互作用。
通过用 $\Exp()$ 将组合结合,我们得到(右)加号算子,
%
\begin{align*}
\SE(n)&:& 
\bfM\oplus\bftau &= \begin{bmatrix}
\bfR\Exp(\bth) & \bft+\bfR\bfV(\bth)\bfrho \\
\bf0 & 1
\end{bmatrix} 
\\
T(n) \tcross \SO(n)&:&
\bfM\oplus\bftau &= \begin{bmatrix}
\bfR\Exp(\bth) & \bft+\bfR\bfp \\
\bf0 & 1
\end{bmatrix} 
\\ 
\langle\bbR^n,\SO(n)\rangle&:&
\bfM\dplus\bftau 
& 
= 
\begin{bmatrix}
\bft+\bfp \\
\bfR\Exp(\bth)
\end{bmatrix} 
\end{align*}
%
其中 $\oplus$ 可用于系统动力学,例如运动积分,但通常不用 $\dplus$,后者可用于对扰动进行建模。
%
他们各自的减号算子读取,
%. 
\begin{align*}
\SE(n)&:&
\bfM_2\ominus\bfM_1 & = \begin{bmatrix}
\bfV_1\inv\bfR_1\tr(\bfp_2 - \bfp_1) \\ \Log(\bfR_1\tr\bfR_2)
\end{bmatrix} 
\\
T(n) \tcross \SO(n)&:&
\bfM_2\ominus\bfM_1 & = \begin{bmatrix}
\bfR_1\tr(\bfp_2 - \bfp_1) \\ \Log(\bfR_1\tr\bfR_2)
\end{bmatrix} 
\\
\langle\bbR^n,\SO(n)\rangle&:&
\bfM_2\dminus\bfM_1 
&= 
\begin{bmatrix}
\bfp_2 - \bfp_1 \\ \Log(\bfR_1\tr\bfR_2)
\end{bmatrix} 
~,
\end{align*}
%
现在这里有趣的是, $\dminus$ 可以用来评估误差和不确定性。这使得 $\dplus,\dminus$ 成为计算导数和协方差的有价值的算子。
\end{fexample}
%%%%%%%%%%%%%%%%%%%%%%%%%%%%%%%%%%%%%%%%%%%%%%%%%%
