% !TEX root = micro_Lie_theory.tex

%%%%%%%%%%%%%%%%%%%% SE3 adjoint %%%%%%%%%%%%%%%%%%%%%%%%%%%%
\begin{fexample}{$\SE(3)$ 的伴随矩阵}\label{ex:SE3_adjoint}
%
刚体运动的 $\SE(3)$ 群(参见 \appRef{sec:SE3})有群,Lie代数和向量元素,
%
\begin{align*}
\bfM&=\begin{bmatrix}\bfR&\bft\\\bf0&1\end{bmatrix}
~,
&
\bftau\hhat&=\begin{bmatrix}\hatx{\bth}&\bfrho\\\bf0&0\end{bmatrix}
~,
&
\bftau &=\begin{bmatrix}\bfrho\\\bth\end{bmatrix}
~.
\end{align*}
%
伴随矩阵由扩展的方程 \eqRef{equ:Adj4} 标识为
%
\begin{align*}
\Ad{\bfM}\,\bftau
  &= (\bfM\bftau\hhat\bfM\inv)\vvee = \cdots =
  \\
  &= \left(\begin{bmatrix}\bfR\hatx{\bth}\bfR\tr & -\bfR\hatx{\bth}\bfR\tr\bft + \bfR\bfrho \\ \bf0 & \bf0\end{bmatrix}\right)\vvee \\
%  &= \left(\begin{bmatrix}\hatx{\bfR\bth} & -\hatx{\bfR\bth}\bft + \bfR\bfrho \\ \bf0 & \bf0\end{bmatrix}\right)\vvee \\
  &= \left(\begin{bmatrix}\hatx{\bfR\bth} & \hatx{\bft}\bfR\bth + \bfR\bfrho \\ \bf0 & \bf0\end{bmatrix}\right)\vvee \\
  &= \begin{bmatrix}\hatx{\bft}\bfR\bth + \bfR\bfrho \\ \bfR\bth\end{bmatrix} 
%  \\
%  &
  = \begin{bmatrix}\bfR & \hatx{\bft}\bfR\\ \bf0&\bfR\end{bmatrix}\begin{bmatrix}\bfrho \\ \bth\end{bmatrix} 
\end{align*}
%
其中我们使用 $\hatx{\bfR\bth}=\bfR\hatx{\bth}\bfR\tr$ 和 $\hatx{\bfa}\bfb=-\hatx{\bfb}\bfa$ 。所以伴随矩阵是 
%
\begin{align*}
\Ad{\bfM} =  \begin{bmatrix}\bfR & \hatx{\bft}\bfR \\ \bf0&\bfR\end{bmatrix} \quad\in\bbR^{6\times6}
~.
\end{align*}
%
\end{fexample}		
%%%%%%%%%%%%%%%%%%%%%%%%%%%%%%%%%%%%%%%%%%%%%%%%%%%%%%%%%%%%%%%%%
