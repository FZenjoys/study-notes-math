% !TEX root = micro_Lie_theory.tex

%%%%%%%%%%%%%%%%%%%% S3 cont %%%%%%%%%%%%%%%%%%%%%%%%%%%%%%%%%
\begin{fexample}{单位四元数群 $S^3$ (条件)}
\label{ex:S3}
在 $S^3$ 群中(回想 \exRef{ex:S3_intro} 并参见 \eg\ \cite{SOLA-17-Quaternion}),单位范数条件 $\bfq^*\bfq=1$ 的时间导数产生 
%
$$\bfq^*\dot\bfq=-(\bfq^*\dot\bfq)^*
.
$$
%
这表明 $\bfq^*\dot\bfq$ 是一个纯虚四元数(其实部为零)。
%The set of pure quaternions is noted $\bbH_p$.
纯虚四元数 $\bfu v\in\bbH_p$ 有形式 
%
$$\bfu v=(iu_x+ju_y+ku_z)v =iv_x+jv_y+kv_z 
,$$
%
其中, $\bfu\te iu_x+ju_y+ku_z$ 是纯幺正的, $v$ 是范数,并且 $i,j,k$ 是Lie代数 $\frak{s}^3=\bbH_p$ 的生成元。
%
重写我们上面的条件,
%
\begin{align*}
\dot\bfq=\bfq\, \bfu v &&\in \mtanat{S^3}{\bfq}
,
\end{align*}
% 
那将积分到 $\bfq = \bfq_0\exp(\bfu v t)$。 
令 $\bfq_0=1$ 并定义 $\bphi \te \bfu\phi \te\bfu v t$ 我们得到指数映射,
%
\begin{align*}
\bfq = \exp(\bfu\phi) \te \sum \frac{\phi^k}{k!}\bfu^k &&\in S^3
~.
\end{align*}
%
$\bfu$ 的幂跟随以下模式 $1,\bfu,-1,-\bfu,1,\cdots$。
因此,我们将项分组为 $1$ 和 $\bfu$ ,并标识 $\cos\phi$ 和 $\sin\phi$ 的级数。
我们得到封闭形式,
%
\begin{align*}
\bfq = \exp(\bfu\phi) = \cos(\phi) + \bfu\sin(\phi) 
~,
\end{align*}
%
这是欧拉公式的优美扩展, $\exp(i\phi)=\cos\phi+i\sin\phi$。
%
Lie代数 $\bfphi=\bfu \phi\in\frak{s}^3$ 的元素可以通过映射 \emph{hat} 和 \emph{vee} 用旋转向量 $\bth\in\bbR^3$ 标识,
%
\begin{align*}
\textrm{Hat}&:& \bbR^3&\to\frak{s}^3;& \bth&\mapsto\bth^\wedge=2\bphi
\\
\textrm{Vee}&:& \frak{s}^3&\to\bbR^3;& \bphi&\mapsto\bphi^\vee=\bth/2
~,
\end{align*}
%
其中 %$\bth \te \bfu\theta\te\bw t$ is the angle-axis vector, and 
因子 $2$ 说明了旋转作用中四元数的双重效应, $\bfx'=\bfq\,\bfx\,\bfq^*$。 
%
通过选择 Hat 和 Vee,四元数指数
%
\begin{align*}
\bfq = \Exp(\bfu\theta)=\cos(\theta/2)+\bfu\sin(\theta/2)
\end{align*}
%
等价于旋转矩阵 $\bfR=\Exp(\bfu\theta)$。
\end{fexample}
%%%%%%%%%%%%%%%%%%%%%%%%%%%%%%%%%%%%%%%%%%%%%%%%%%%%%%%%
