% !TEX root = micro_Lie_theory.tex

%%%%%%%%%%%%%%%%%%%%%%%%%%% SO3 %%%%%%%%%%%%%%%%%%%%%%%%%%%%%%%%%%
\begin{fexample}{旋转群 $\SO(3)$, 它的 Lie 代数 $\so(3)$, 及其向量空间 $\bbR^3$}
\label{ex:SO3}
在 $3\tcross3$ 旋转矩阵 $\bfR$ 的旋转群 $\SO(3)$ 中,我们有正交条件 $\bfR\tr\bfR=\bfI$。
切空间可以通过取该约束的时间导数来找到,即 $\bfR\tr\dot\bfR+\dot\bfR\tr\bfR=0$,我们将其重新排列为 $$\bfR\tr\dot\bfR=-(\bfR\tr\dot\bfR)\tr.$$
这个表达式表明 $\bfR\tr\dot\bfR$ 是一个斜对称矩阵(其转置为其负值)。
斜对称矩阵通常用 $\hatx{\bw}$ 表示,其形式如下 
%
$$\hatx{\bw}=\begin{bsmallmatrix}
0 & -\omega_z & \omega_y \\
\omega_z & 0 & -\omega_x \\
-\omega_y & \omega_x & 0 
\end{bsmallmatrix}
.$$
%
这给出 $\bfR\tr\dot\bfR=\hatx{\bw}$。当 $\bfR=\bfI$ 我们有 $$\dot\bfR=\hatx{\bw},$$ 也就是说 $\hatx{\bw}$ 在 $\SO(3)$ 的Lie代数中,我们称之为 $\so(3)$。
%
由于 $\hatx{\bw}\in\so(3)$ 有 $3$ 个自由度, $\SO(3)$ 的维数是 $m=3$。
Lie代数是一个向量空间,它的元素可以分解为
%
$$
\hatx{\bw} = 
  \omega_x\bfE_x+
  \omega_y\bfE_y+
  \omega_z\bfE_z
$$
其中  
$
\bfE_x=
\begin{bsmallmatrix}0&0&0\\0&0&-1\\0&1&0\end{bsmallmatrix}$, 
$\bfE_y=
\begin{bsmallmatrix}0&0&1\\0&0&0\\-1&0&0\end{bsmallmatrix}$,
 $\bfE_z=
\begin{bsmallmatrix}0&-1&0\\1&0&0\\0&0&0\end{bsmallmatrix}$ 
为 $\so(3)$ 的生成元,并且其中 $\bw=(\omega_x,\omega_y,\omega_z)\in\bbR^3$ 是角速度向量。 
%
上面的一对一线性关系允许我们用 $\bbR^3$ 标识 $\so(3)$ --- 我们写为 $\so(3)\cong\bbR^3$。
我们使用线性算子 \emph{hat} 和 \emph{vee} 从 $\so(3)$ 传递到 $\bbR^3$ ,反之亦然,
%
\begin{align*}
\textrm{Hat}&:& \bbR^3&\to\so(3);& \bw&\mapsto\bw^\wedge=\hatx{\bw} 
\\
\textrm{Vee}&:& \so(3)&\to\bbR^3;& \hatx{\bw}&\mapsto\hatx{\bw}^\vee=\bw
~.
\end{align*}
\end{fexample}
%%%%%%%%%%%%%%%%%%%%%%%%%%%%%%%%%%%%%%%%%%%%%%%%%%%%%%%%%%%
